% !TeX document-id = {95a8edac-103b-4a15-9950-88e55882fea3}
% !TeX program = xelatex

\documentclass[aspectratio=169,xcolor={dvipsnames}
%,notes=only
%,notes
%,show notes on second screen=right
%,handout
]{beamer}
\usetheme[background=light, numbering=fraction]{metropolis}
\usepackage{appendixnumberbeamer}
\usepackage{pgfpages}

\usepackage{FiraMono}

\setmonofont[
Scale=MatchLowercase,
Contextuals={Alternate}
]{Fira Mono}

\usepackage[newfloat,cache=false]{minted}
% \usemintedstyle{autumn}
\usemintedstyle{tango}

\setminted[rust]{obeytabs=true,tabsize=2}
\setminted[toml]{obeytabs=true,tabsize=2}

\usepackage{twemojis}

% Helpful defines
\newcommand{\ourtech}{\textsc{Galette}}
\newcommand{\afxdp}{\texttt{AF\_XDP}}
\newcommand{\af}{(\texttt{AF\_})XDP}
\newcommand{\afp}{\texttt{AF\_PACKET}}

% You will need to modify these, the authors down below,
% and likely the \addbibresource statements below.
\newcommand{\mytitle}{\ourtech:~a~Lightweight~XDP~Dataplane on~your~Raspberry~Pi}
\newcommand{\myemail}{kylesimpson1@acm.org}
\newcommand{\myurl}{https://mcfelix.me}
\newcommand{\mygithub}{FelixMcFelix}

%\usepackage[T1]{fontenc}

\usepackage{bm}
\usepackage{mathtools}

\usepackage[labelfont=bf,textfont={it}]{caption}
\usepackage{subcaption}
\captionsetup[figure]{justification=centering}
\captionsetup[subfigure]{justification=centering}

\usepackage{tikz}
\usepackage{varwidth}
\usetikzlibrary{arrows.meta, calc, fit, positioning, shapes}

\usepackage[title]{appendix}

\usepackage{etoolbox}
\usepackage[per-mode=symbol]{siunitx}
\robustify\bfseries
\robustify\emph
%\robustify\uline
\sisetup{detect-all, range-phrase=--, range-units=single, detect-weight=true, table-format=1.3}
\DeclareSIUnit{\packet}{p}

\usepackage[siunitx]{circuitikz}

%\usepackage{fontspec}
%\setsansfont{Fira Sans Mono}

\usepackage[UKenglish]{babel}
\usepackage{csquotes}

\usepackage{amssymb}

\usepackage{lipsum}
%\usepackage[basic]{complexity}
\usepackage[super,negative]{nth}

\usepackage{booktabs}

%bib
\usepackage[maxnames=3,maxbibnames=99,mincrossrefs=5,sortcites
%,backend=bibtex
,style=authortitle
]{biblatex}
\addbibresource{bibliography.bib}

% official colours
\definecolor{uofguniversityblue}{rgb}{0, 0.219608, 0.396078}

\definecolor{uofgheather}{rgb}{0.356863, 0.32549, 0.490196}
\definecolor{uofgaquamarine}{rgb}{0.603922, 0.72549, 0.678431}
\definecolor{uofgslate}{rgb}{0.309804, 0.34902, 0.380392}
\definecolor{uofgrose}{rgb}{0.823529, 0.470588, 0.709804}
\definecolor{uofgmocha}{rgb}{0.709804, 0.564706, 0.47451}

\definecolor{uofglawn}{rgb}{0.517647, 0.741176, 0}
\definecolor{uofgcobalt}{rgb}{0, 0.615686, 0.92549}
\definecolor{uofgturquoise}{rgb}{0, 0.709804, 0.819608}
\definecolor{uofgsunshine}{rgb}{1.0, 0.862745, 0.211765}
\definecolor{uofgpumpkin}{rgb}{1.0, 0.72549, 0.282353}
\definecolor{uofgthistle}{rgb}{0.584314, 0.070588, 0.447059}
\definecolor{uofgpillarbox}{rgb}{0.701961, 0.047059, 0}
\definecolor{uofglavendar}{rgb}{0.356863, 0.301961, 0.580392}

\definecolor{uofgsandstone}{rgb}{0.321569, 0.278431, 0.231373}
\definecolor{uofgforest}{rgb}{0, 0.317647, 0.2}
\definecolor{uofgburgundy}{rgb}{0.490196, 0.133333, 0.223529}
\definecolor{uofgrust}{rgb}{0.603922, 0.227451, 0.023529}

\definecolor{inferno0}{rgb}{0.001462 0.000466 0.013866}
\definecolor{inferno64}{rgb}{0.341500 0.062325 0.429425}
\definecolor{inferno128}{rgb}{0.735683 0.215906 0.330245}
\definecolor{inferno192}{rgb}{0.978422 0.557937 0.034931}
\definecolor{inferno255}{rgb}{0.988362 0.998364 0.644924}

%picky abt et al.
\usepackage{xpatch}

\makeatletter\let\expandableinput\@@input\makeatother

\xpatchbibmacro{name:andothers}{%
	\bibstring{andothers}%
}{%
	\bibstring[\emph]{andothers}%
}{}{}

%opening!

\usepackage{cleveref}
\newcommand{\crefrangeconjunction}{--}

\usepackage{fontawesome5}

\addtobeamertemplate{footnote}{\vspace{-6pt}\advance\hsize-0.5cm}{\vspace{6pt}}
\makeatletter
% Alternative A: footnote rule
\renewcommand*{\footnoterule}{\kern -3pt \hrule \@width 2in \kern 8.6pt}
% Alternative B: no footnote rule
% \renewcommand*{\footnoterule}{\kern 6pt}
\makeatother

\usepackage[export]{adjustbox}
\usetikzlibrary{arrows.meta, calc, fit, positioning, shapes.misc}

% \expandableinput is useful for some tables etc.
\makeatletter\let\expandableinput\@@input\makeatother
\newcommand{\cmark}{\ding{51}}%
\newcommand{\xmark}{\ding{55}}%

%-------------------------------------%
%-------------------------------------%

\title{\mytitle}
\author{\vspace{-1em}\textbf{Kyle A.\ Simpson}, Chris Williamson, Douglas J.\ Paul, Dimitrios P.\ Pezaros\\
	\faEnvelopeOpen{} \href{mailto:\myemail}{\nolinkurl{\myemail}}\\
	\vspace{1em}\small{\faGithub{} \href{https://github.com/\mygithub}{\mygithub} \hspace{0.5em} \faGlobe{} \url{\myurl}}}
\institute{University of Glasgow}
\date{13th June, 2023}

\begin{document}
% title fun, including Org logos....
\begin{frame}
	\maketitle
	\begin{tikzpicture}[overlay, remember picture]
%		\node[above right=0.8cm and 0.9cm of current page.south west] (esnet-logo) {\includegraphics[width=2.75cm]{netlab-trim}};
%		\node[right=1cm of esnet-logo] {\adjincludegraphics[height=2cm,trim={0 {.4\height} 0 {.05\height}},clip]{uofg}};
		\node[above right=-0.1cm and 0.8cm of current page.south west] (uofg-logo) {\adjincludegraphics[height=2cm,trim={0 {.4\height} 0 {.05\height}},clip]{branding/uofg}};
		\node[right=0.5cm of uofg-logo] {\includegraphics[width=2.75cm]{branding/netlab-trim}};
	\end{tikzpicture}
\end{frame}

\begin{frame}{Securing Sensor \& IoT Gateway Networks}
	\begin{columns}
		\begin{column}{0.6\linewidth}
%			\alert{A thing}: bright results!
			\begin{itemize}
				\item Security -- ingress/egress packet processing by \emph{network functions}.
				\begin{itemize}
					\item \alert{IP layer} -- Firewalls, DPI, ACLs...
					\item Middleboxes a bad fit.
					\item Needs to be \alert{reconfigurable} -- attacks and security context evolve.
				\end{itemize}
				\item Ideally \alert{in-situ}.
				\begin{itemize}
					\item Dynamic/retrofitted.
					\item But limited space + power in the field.
					\item Physically vulnerable!
				\end{itemize}
			\item \emph{Sensor networks have low data rates! }
			\end{itemize}
		\end{column}
		\begin{column}{0.4\linewidth}
			\begin{figure}
				\includegraphics[keepaspectratio,width=\linewidth]{images/rpi-sens}
			\end{figure}
		\end{column}
	\end{columns}
\end{frame}

\begin{frame}{Fast, cheap, and secure IoT Defence -- pick 3?}
	\begin{columns}
		\begin{column}{0.4\linewidth}
			\begin{figure}
				\includegraphics[keepaspectratio,width=\linewidth]{images/rpi}
			\end{figure}
		\end{column}
		\begin{column}{0.6\linewidth}
			%			\alert{A thing}: bright results!
			\begin{itemize}
				\item Single-board compute like RPis are small, capable, affordable! \alert{Cheap!}
				\begin{itemize}
					\item See also: NUCs (££), Jetsons (£££).
					\item \emph{Linux-based}: Easy(/ier) to target and write for. \alert{We also get kernel network stack advancements.}
					\item \textbf{Different CPU architectures.}
				\end{itemize}
				\item Project goals:
				\begin{itemize}
					\item \alert{Fast!} Low-latency, quickly reconfigurable.
					\item \alert{Secure!} efficient NFV code gen from \emph{memory-safe languages}.
				\end{itemize}
			\end{itemize}
		\end{column}
	\end{columns}
\end{frame}

\begin{frame}{\ourtech{}'s Research Objectives}
	\alert{\textbf{\emph{\ourtech{}}} puts \textbf{effective} eBPF packet processing into \textbf{edge computers}.}
	
	\begin{enumerate}
		\item What \emph{specialisations} does XDP Function Chaining need to \emph{best suit SBCs}?
		\begin{itemize}[<+->]
			\item Split userland-XDP pipeline.
			\item \alert{Many userland pipes!}
		\end{itemize}
		\item How do we make eBPF + native compile from memory-safe systems languages easy? And portable across `native'?
		\begin{itemize}[<+->]
			\item \alert{One Rust program per NF} $\implies$ eBPF + native.
			\item Easier, unified API.
			\item Simple, dynamic chain format.
		\end{itemize}
		\item How efficient is it on RPi/NUC?
		\begin{itemize}[<+->]
			\item Better latency, throughout, power use than \afp{}...
			\item \alert{\textbf{...without polling.}}
		\end{itemize}
	\end{enumerate}
%	\begin{itemize}
%%		\item ?? 3 things in my work now: how do we make joint compile in secure, memsafe lang easier? How do we specialize xdp sfc for cheaper devices? How much better is it [power, perf, lat]?
%		
%		\item Fast reconfiguration:
%		\begin{itemize}
%			\item State, Program Code, \alert{Composition}
%		\end{itemize}
%%		\item Attestation and authentication:
%%		\begin{itemize}
%%			\item Right programs on right machine, requested by trusted server.
%%		\end{itemize}
%		\item `Acceptably' low-latency packet-processing, without pushing CPU/power draw too high?
%		\begin{itemize}
%			\item I.e., as low as we can get without polling.
%		\end{itemize}
%		\item Easy development and composition.
%		\begin{itemize}
%			\item One Rust program per NF $\implies$ compiled for stack.
%			\item Simple, dynamic chain format.
%		\end{itemize}
%	\end{itemize}
\end{frame}

\section{Background}

%\begin{frame}{Limits of existing SFC}
%	\begin{itemize}
%		\item `Best' low latency processing (DPDK) is \alert{expensive} -- CPU and power.
%		\begin{itemize}
%			\item ...IFF you have HW support (NUCs)
%		\end{itemize}
%%		\item SotA in \emph{secure} processing needs server-only capabilities like \emph{trusted execution environments} (TEEs).
%		\item No powerful hardware offloads or acceleration.
%		\begin{itemize}
%			\item FPGA hats/daughterboards \alert{`off-path'}
%		\end{itemize}
%		\item Devices physically vulnerable, \alert{no ECC memory}.
%		\item ...So, how to reconcile with cheap \& portable SBCs?
%	\end{itemize}
%\end{frame}

%\begin{frame}{What tools do we \emph{consistently} have?}
%	\begin{itemize}
%		\item SBCs often linux-based
%		\begin{itemize}
%			\item Easy(/ier) to target and write for.
%			\item \alert{Advantage:} We also get kernel network stack advancements.
%		\end{itemize}
%		\item Can run commodity software with no issues, reasonable target archs like Aarch64, x86\_64, ...
%		\item Includes, principally, eBPF tooling!
%%		\item XDP since 2016, \texttt{AF\_XDP} since 2019 (kver 4.18). more features in newer kernels! (recent: (\texttt{AF\_})XDP for windows)
%%		\item Key to XDP's value: stack bypass that \emph{improves} with driver support vs. builtin.
%	\end{itemize}
%\end{frame}

%\section{Hm?}

\begin{frame}{eBPF: What and Why?}
	\begin{columns}
		\begin{column}{0.6\linewidth}
			\begin{itemize}
				\item Simple register machine VM (user-written) code, derived from BPF.
				\item Modern use -- Kernel hooks, perf instrumentation, debugging
				\item JIT compiled
				\item Kernel-verified
				\begin{itemize}
					\item Bounds-checked pointer accesses
					\item Program size limited, no unbounded loops
					\item Syscalls (\emph{eBPF helpers}) exposed based on hook point
				\end{itemize}
			\end{itemize}
		\end{column}
		\begin{column}{0.4\linewidth}
			\begin{figure}
				\centering
				\includegraphics[width=0.9\linewidth,keepaspectratio]{images/ebpf}
			\end{figure}
		\end{column}
	\end{columns}
\end{frame}

\begin{frame}{Network stack improvements: XDP}
	\begin{columns}
		\begin{column}{0.5\linewidth}
			\resizebox{\linewidth}{!}{
%				\include{diagrams/offloading}
\colorlet{ol-phys}{uofgforest}
\colorlet{ol-log}{uofglavendar}
\colorlet{ol-user}{uofgrust}
\colorlet{ol-userland}{ol-user!75}

\colorlet{ol-arrow}{black}
\colorlet{ol-user-arrow}{ol-user}

\begin{tikzpicture}
	\draw[color=ol-phys,fill=ol-phys!10] (0,0) rectangle ++(2,1) node[pos=.5] (mac) {MAC};
	\draw[color=ol-phys,fill=ol-phys!10] ($(mac) + (2, -0.5)$) rectangle ++(2,1) node[pos=.5] (nic) {NIC};
	\draw[color=ol-phys,fill=ol-phys!10,align=center,text=black] ($(nic) + (2, -0.5)$) rectangle ++(2,1) node[pos=.5] (mem) {Memory\\\& Cache};
	
	\draw[color=ol-log,fill=ol-log!10,align=center,text=black] ($(mem) + (2, -0.5)$) rectangle ++(2,1) node[pos=.5] (rx-tx) {Driver\\Rx/Tx};
	\draw[color=ol-log,fill=ol-log!10,align=center,text=black] ($(rx-tx) + (-1, -2.5)$) rectangle ++(2,1) node[pos=.5] (skb) {Kernel\\SKB Alloc};
	\draw[color=ol-log,fill=ol-log!10,align=center,text=black] ($(skb) + (-1, -2.5)$) rectangle ++(2,1) node[pos=.5] (ns) {Network\\Stack};
	
	\draw[color=ol-userland, fill=ol-userland,align=center, text=white, rounded corners] ($(ns) + (-5, -0.5)$) rectangle ++(2,1) node[pos=.5] (userland) {Userland\\Code};
	
	\draw[color=ol-user, fill=ol-user,align=center,text=white, rounded corners] ($(nic) + (-2, -3.5)$) rectangle ++(2,1) node[pos=.5] (smartnic-offload) {Offload\\C/P4/eBPF};
	
	\draw[color=ol-user, fill=ol-user,align=center,text=white, rounded corners] ($(skb) + (-5, -0.5)$) rectangle ++(2,1) node[pos=.5] (xdp-offload) {Offload\\eBPF (XDP)};
	
	% --------
	
	\node[color=ol-phys] at (0.75, 1.35) {\large{}Physical};
	\node[color=ol-log, rotate=270] at (11.5, 0.5) {\large{}Logical};
	
	% --------
	
	\draw[<->, thick, color=ol-arrow] (mac) -- (nic);
	\draw[<->, thick, color=ol-arrow] (nic) -- (mem);
	\draw[<->, thick, color=ol-arrow] (mem) -- (rx-tx) node[midway, above] {\small{}IRQs};
	\draw[<->, thick, color=ol-arrow] (rx-tx) -- (skb);
	\draw[<->, thick, color=ol-arrow] (skb) -- (ns);
	
	
	\draw[<->, thick, color=ol-user-arrow, shorten >=0.25cm,shorten <=0.3cm] (ns) -- (userland) node[midway, below] {\small{}Socket};
	\draw[<->, thick, color=ol-user-arrow, shorten >=0.12cm,shorten <=0.17cm] (nic) -- (smartnic-offload) node[midway, left, align=center] {\small{}SmartNIC\\Offload};
	\draw[<->, thick, color=ol-user-arrow, shorten >=0.12cm,shorten <=0.12cm] (rx-tx) -- (xdp-offload) node[midway, above, sloped] {\small{}Native \color{ol-user-arrow}XDP};
	\draw[<->, thick, color=ol-user-arrow, shorten >=0.05cm,shorten <=0.12cm] (skb) -- (xdp-offload) node[midway, above] {\small{}Generic \color{ol-user-arrow}XDP};
	\draw[<->, thick, color=ol-user-arrow, shorten >=0.12cm,shorten <=0.12cm] (userland) -- (xdp-offload) node[midway, right] {\small{}\texttt{AF\_XDP}};
	
	\draw[<->, thick, color=ol-user-arrow, shorten >=0.12cm,shorten <=0.12cm, out=200,in=150] (mem) to node[left] {\small{}DPDK} (userland);
\end{tikzpicture}
			}
		\end{column}
		\begin{column}{0.5\linewidth}
			\begin{itemize}
				\item eBPF hook attached to \alert{packet ingress}
				\item Variations on hook $\in \left\{\text{Offload, Driver, Generic}\right\}$
				\begin{itemize}
					\item Perf degrades gracefully according to driver support
				\end{itemize}
				\item Hook can modify \& inspect packets before forwarding to Linux stack, sending \alert{straight to (another) NIC}, or drop.
				\item Since 2019: \texttt{AF\_XDP} stack bypass!
			\end{itemize}
		\end{column}
	\end{columns}
\end{frame}

\section{Q1: Specialising \afxdp{} Function Chaining for SBCs}

\begin{frame}{The Unique Challenges of SBCs}
	\begin{itemize}
		\item \textbf{Problem:} `Best' low latency processing (DPDK) is \alert{expensive} -- CPU, power, \emph{HW support}.
		\item \textbf{Problem:} Mismatch of HW queues to physical cores:
		\begin{itemize}[<+->]
			\item \alert{Soln:} load balance and place high-latency NFs in userland.
			\item ...also, don't pass packets back to kernel-space.
		\end{itemize}
		\item \textbf{Problem:} XDP hooks only on ingress (\emph{for now}):
		\begin{itemize}[<+->]
			\item \alert{Soln:} Write an individual NF \emph{once}, compile for both envs, and replicate NFs as needed.
		\end{itemize}
	\end{itemize}
\end{frame}

\begin{frame}{\ourtech{} Design: Bird's eye view}
	\begin{columns}
		\begin{column}{0.5\linewidth}
			\begin{figure}
				\centering
				\resizebox{0.9\linewidth}{!}{\hspace{-0.9cm}\input{diagrams/two-tier}\hspace{-0.2cm}}
			\end{figure}
		\end{column}
		\begin{column}{0.5\linewidth}
			\begin{itemize}
				\item Two-tier approach---XDP \& User.
				\item Composable NFs -- graph structure.
				\item Critical or high performance NFs go into XDP:
				\begin{itemize}
					\item \alert{Low latency for most packets}.
					\item Chain with XDP tail calls.
				\end{itemize}
				\item Rare `slow-path' still kernel bypass:
				\begin{itemize}
					\item Expensive \& proprietary code.
					\item Only for candidate attack traffic.
				\end{itemize}
				\item Reconfigurable, dynamic.
				\item \alert{Remote-compiled.}
			\end{itemize}
		\end{column}
	\end{columns}
\end{frame}

\begin{frame}{How does this differ from other frameworks?}
	\textbf{In Security?} SafeBricks~\footcite{DBLP:conf/nsdi/PoddarLPR18}, AuditBox~\footcite{DBLP:conf/nsdi/LiuSKPSS21} or similar.
	\begin{itemize}
		\item ...No SGX support in devices of interest.
	\end{itemize}
	
	\textbf{In eBPF/XDP space?} Polycube~\footcite{DBLP:journals/tnsm/MianoRBBL21}!
	\begin{itemize}
		\item Built around datacentres -- we often have just one HW queue for a NIC.
		\item \alert{...so we use more userland pipes to scale to the extra cores we \emph{do} have.}
	\end{itemize}
\end{frame}

\begin{frame}{How do we upcall to userland?}
	\begin{columns}
		\begin{column}{0.5\linewidth}
			\begin{itemize}
				\item \textbf{Problem:} Can send packet over \texttt{AF\_XDP}, but \emph{no context on what the next (callee) NF is}.
				\begin{itemize}
					\item Polycube's solution inadequate: one discrete userland component per \emph{cube}.
				\end{itemize}
				\item \alert{Soln:} Adjust headroom of packets, write in ID and action of caller.
%				\item ...might be a \texttt{memcpy}, but ideally only paid on packets who need it -- dpolier??there isn't.
%				\item ?? Diagram from paper here?
				
			\end{itemize}
		\end{column}
		\begin{column}{0.5\linewidth}
			\begin{figure}
				\centering
				\resizebox{0.9\linewidth}{!}{\tikzset{
	nfbox/.style={draw,rounded corners,color=uofgsandstone,fill=uofgsandstone!5},
	membox/.style={draw,color=uofgsandstone,fill=uofgsandstone!5},
	idbox/.style={draw,color=uofgpillarbox,fill=uofgpillarbox!5},
	actbox/.style={draw,color=uofgpumpkin,fill=uofgpumpkin!5},
	authflow/.style={color=uofgthistle,thick},
	putflow/.style={color=uofgpillarbox,thick,dash dot},
	readflow/.style={color=uofgcobalt,thick,dash dot},
	zoomflow/.style={color=uofgcobalt!50,thick,dotted},
}

\begin{tikzpicture}
\node (rxq) {\includegraphics[keepaspectratio,width=3em]{images/ringbuf}};
\node (rxq-l) at ($(rxq) + (0,-0.75)$) {Rx};

\node (txq) at ($(rxq) + (0,-1.5)$) {\includegraphics[keepaspectratio,width=3em]{images/ringbuf-alt}};
\node (txq-l) at ($(txq) + (0,-0.75)$) {Tx};

\node (cq) at ($(txq) + (0,-1.5)$) {\includegraphics[keepaspectratio,width=3em]{images/ringbuf-alt2}};
\node (cq-l) at ($(cq) + (0,-0.75)$) {Completion};

\node (umem) at ($(rxq) + (3.5,0)$) {
	\begin{tikzpicture}
		\node[rotate=90] (umem-label) at (0,0) {\textsc{Umem}};
		\draw[membox, opacity=0.5] ($(umem-label) + (0.25,0.25)$) rectangle ++(4,0.25);
		\draw[membox] ($(umem-label) + (0.25,0)$) rectangle ++(4,0.25);
		\draw[membox, opacity=0.5] ($(umem-label) + (0.25,-0.25)$) rectangle ++(4,0.25);
		\draw[membox, opacity=0.5] ($(umem-label) + (0.25,-0.5)$) rectangle ++(4,0.25);
		
		\draw[actbox] ($(umem-label) + (0.25,0)$) rectangle ++(1.5,0.25);
		\draw[idbox] ($(umem-label) + (0.25,0)$) rectangle ++(0.75,0.25);
		
		\node[color=uofgpillarbox] at ($(umem-label) + (0.62,0.125)$) {\scriptsize{NF ID}};
		\node[color=uofgpumpkin!50!uofgpillarbox] at ($(umem-label) + (1.38,0.125)$) {$\circledast$};
		\node[color=uofgsandstone] at ($(umem-label) + (3,0.11)$) {\scriptsize{Packet Body}};
	\end{tikzpicture}
};

\node[draw, rectangle, align=left] (b1) at (3.5,-1) {Get packet(s)};
\node[draw, rectangle, align=left] (b2) at ($(b1) + (0,-0.8)$) {\texttt{next\_nf(ID,$\circledast$)}};
\node[draw, rectangle, align=left] (b3) at ($(b2) + (0,-0.8)$) {\texttt{nf\_dylib(pkt,$\triangle$)}};
\node[draw, rectangle, align=left] (b4) at ($(b3) + (0,-0.8)$) {Cleanup};

\node (nfs) at (7,-2.5) {
	\begin{tikzpicture}
		\draw[nfbox, opacity=0.25] (0.2,-0.2) rectangle ++(1,1);
		\draw[nfbox, opacity=0.5] (0.1,-0.1) rectangle ++(1,1);
		\draw[nfbox] (0,0) rectangle ++(1,1);
		
		\node at (0.5,0.4) {\faMicrochip};
		\node at (0.35,0.85) {\scriptsize{\textbf{NF \emph{b}}}};
	\end{tikzpicture}
};

\draw[-,zoomflow] (0.22,-0.18) -- (1.4,0.24);
\draw[-,zoomflow] (0.4,-0.24) -- (1.4,0);

\draw[->, authflow] (b1.east) to[out=-10,in=90] (b2.north east);
\draw[->, authflow] (b2.east) to[out=-10,in=10] node[midway, right] {$\triangle$} (b3.east);
\draw[->, authflow] (b3.south west) to[out=-180,in=180, parabola height=1cm] node[midway, left, sloped, yshift=0.25cm, xshift=0.75cm] {\texttt{Call:} $\triangle'$} (b3.north west);
\draw[->, authflow] (b3.south east) to[out=-30,in=10, parabola height=1cm] (b4.east);

\draw[->, authflow, opacity=0.25] (b4.west) to[out=-180,in=-180, parabola height=-5cm] (b1.west);

\draw[->, readflow] (rxq) to node[midway, below, xshift=-0.4cm] {\footnotesize\texttt{read()}} (b1.west);
\draw[->, readflow] (nfs) to node[midway, below] {\footnotesize\texttt{get($\triangle$)}} (b3.east);

\draw[->, putflow] (b4) to[out=-170, in=-30] node[midway, below] {\footnotesize\texttt{Drop}} (cq);
\draw[->, putflow] ($(b3.west) + (-0.55,0)$) to node[midway, below] {\footnotesize\texttt{Tx}} (txq);
\end{tikzpicture}}
				\caption{Packet processing in the XDP Fast Path (NF maps omitted).\label{fig:dplane-xdp}}
			\end{figure}
		\end{column}
	\end{columns}
\end{frame}

\section{Q2: Easy Joint-Compile (eBPF + Native) from Rust \twemoji{crab}}

\begin{frame}[fragile=singleslide]{Skeleton details}
	\begin{columns}
		\begin{column}{0.5\linewidth}
			\begin{itemize}
				%				\item ?? Discuss API etc here in more detail?
				\item Consistent NF API for both XDP/userland.
				\item Rust compiler should be able to enforce...
				\begin{itemize}
					\item \mintinline{rust}|#![forbid(unsafe_code)]| (or similar cargo tooling) on NF module crates,
					\item all NF branches specified.
				\end{itemize}
				\item All compilation on external server.
				\begin{itemize}
					\item SBC too constrained.
					\item If compile-server is TEE-equipped, \alert{can attest compiler/code} etc. following SotA!
				\end{itemize}
			\end{itemize}
		\end{column}
		\begin{column}{0.5\linewidth}
			\centering
%			\begin{minted}[fontsize={\fontsize{5.5}{6.5}\selectfont}]{rust}
%#![no_std]
%pub enum Action {
%	Left,
%	Right,
%	Up,
%	Down,
%}
%				
%pub fn packet(bytes: impl Packet) -> Action {
%	let addr_lsb_idx = 14 +
%	match pkt.slice_from(12, 2) {
%		Some(&[0x08, 0x00]) => 19, //v4
%		Some(&[0x86, 0xDD]) => 39, //v6
%		_ => {return Action::Left},
%	};
%	
%	match pkt.slice_from(addr_lsb_idx, 1)
%		.map(|v| v[0] % 2) {
%			Some(0) => Action::Left,
%			Some(1) => Action::Right,
%			Some(2) => Action::Up,
%			Some(3) => Action::Down,
%			_ => unreachable!(),
%	}
%}
%			\end{minted}
\inputminted[fontsize={\fontsize{5.5}{6.5}\selectfont}]{rust}{listings/macswap.rs}
			\textbf{\texttt{mod.rs}: A counting macswap function}
		\end{column}
	\end{columns}
\end{frame}

\begin{frame}[fragile=singleslide]{A Service Funtion Chain: \texttt{security.toml}}
	\begin{columns}
		\begin{column}{0.5\linewidth}
			\begin{minted}[fontsize={\fontsize{7.5}{8.5}\selectfont}]{toml}
				# -- NF & Map definitions --
				[functions.access-control.maps]
				allow-list = {
					type = "lpm-trie",
					size = 65535
				}
				
				[functions.weak-classifier]
				maps = { flow-state = "_" }
				
				[functions.dpi]
				maps = { flow-state = "_" }
				disable_xdp = true
				
				[maps.flow-state]
				type = "hash_map"
				size = 65535
			\end{minted}
		\end{column}
		\begin{column}{0.5\linewidth}
			\begin{minted}[fontsize={\fontsize{7.5}{8.5}\selectfont}]{toml}
				# -- Chain definition --
				[[links]]
				from = "rx"
				to = ["access-control"]
				
				[[links]]
				from = "access-control"
				to = ["tx", "weak-classifier"]
				
				[[links]]
				from = "weak-classifier"
				to = ["tx", "!dpi", "drop"]
				
				[[links]]
				from = "dpi"
				to = ["tx", "drop"]
			\end{minted}
		\end{column}
	\end{columns}
\end{frame}

\begin{frame}[fragile=singleslide]{A Peek Behind The Curtain }
	\begin{columns}
		\begin{column}{0.5\linewidth}
			\begin{minted}[fontsize={\fontsize{8.5}{9.5}\selectfont}]{rust}
pub struct PodData {
	pub a: u8,
	pub b: bool,
	pub c: u64,
}

#[maps]
pub struct TestMaps {
	plain: (u32, u64),
	composite: (u32, PodData),
}
			\end{minted}
		\end{column}
			\hspace{-9em}$\bm\implies$\hspace{2em}
		\begin{column}{0.45\linewidth}
			\begin{minted}[fontsize={\fontsize{7.5}{8.5}\selectfont}]{rust}
pub type NfKeyTy0 = u32;
pub type NfKeyTy1 = u32;
pub type NfValTy0 = u64;
pub type NfValTy1 = PodData;

pub struct TestMaps<NfMapField0, NfMapField1>
where
	NfMapField0: Map<u32, u64>,
	NfMapField1: Map<u32, PodData>,
{
	pub plain: NfMapField0,
	pub composite: NfMapField1,
}
			\end{minted}
		\end{column}
	\end{columns}
	And templating code parses any \mintinline{rust}|struct|s tagged \mintinline{rust}|#[maps]| to count \& \emph{generate output crates!}
\end{frame}

%\begin{frame}{A Peek Behind The Curtain (II)}
%	content...
%\end{frame}


%\begin{frame}{Methodology (I): Low-latency XDP fast-path}
%	\begin{columns}
%		\begin{column}{0.5\linewidth}
%			\begin{figure}
%				\centering
%				\resizebox{0.9\linewidth}{!}{\hspace{-0.9cm}\input{diagrams/two-tier}\hspace{-0.2cm}}
%			\end{figure}
%		\end{column}
%		\begin{column}{0.5\linewidth}
%			\begin{itemize}
%				\item Two-tier approach---XDP \& User.
%				\item Composable NFs -- graph structure.
%				\item Critical or high performance NFs go into XDP:
%				\begin{itemize}
%					\item Early results -- \alert{low latency for most packets}.
%				\end{itemize}
%				\item Rare `slow-path' still kernel bypass:
%				\begin{itemize}
%					\item Expensive \& proprietary code.
%					\item Only for candidate attack traffic.
%				\end{itemize}
%				\item Reconfigurable, dynamic.
%			\end{itemize}
%		\end{column}
%	\end{columns}
%\end{frame}

%\begin{frame}{Control plane: PUF-based authentication}
%	\begin{itemize}
%		\item How to attest the above code and config is correct?
%		\begin{itemize}
%			\item TLS w/ pre-shared certs works well.
%			\item But \alert{corruption, unplanned expiry possible on field devices}.
%		\end{itemize}
%		\item \alert{\emph{Physical Unclonable Functions} (PUFs)} -- input-based device signatures, CRPs.
%		\item Authenticate keys in the wild without root certs.
%		\begin{itemize}
%			\item Two-way: Client $\leftrightarrow$ Server!
%			\item Soln: RusTLS modification to declare challenge via X.509 extension, mix response bits into signature algo input [Zero-knowledge].
%		\end{itemize}
%		\item Strong attestation of identities to physical devices.
%	\end{itemize}
%\end{frame}
%
%\begin{frame}{Control plane: PUF-based authentication (II)}
%	\begin{columns}
%		\begin{column}{0.5\linewidth}
%			\begin{itemize}
%				\item \alert{RTD}-based array designs -- quantum property.
%				\item Behaviour in purple region (NDR region) \alert{physical device-dependent}
%				\begin{itemize}
%					\item Perturbations from `ideal' behaviour can't be replicated
%					\item N\unit{\degree} peaks and perturbations depend on active devices.
%				\end{itemize}
%				\item Challenge bits control used transistors in circuit
%				\begin{itemize}
%					\item $\sim$ Exp amount in $n$, Large Resp.
%				\end{itemize}
%			\end{itemize}
%		\end{column}
%		\begin{column}{0.5\linewidth}
%			\begin{figure}
%				\centering
%				\resizebox{\linewidth}{!}{\hspace{-1cm}\colorlet{challengecolour}{uofgcobalt}
\colorlet{challengemapcolour}{challengecolour!40}

\colorlet{respcolour}{uofgthistle}

\begin{tikzpicture}
	\node (circuit) at (0,0) {\resizebox{4cm}{!}{
		\begin{circuitikz}
%			\draw (0,0) to[short, o-*] (1,0) to [cuteopenswitchshape] (2,0) to[tDo] (3,0);
			
			% main block
			
			\node [cuteopenswitchshape] (sw1) at (2,2) {};
			\draw (sw1.out) to[tDo] ++(1,0) to[short, -*] ++(1,0) -- ++(0,-3);
			
			\node [cuteclosedswitchshape] (sw2) at ($(sw1) - (0,1)$) {};
			\draw (sw2.out) to[tDo] ++(1,0) to[short, -*] ++(1,0);
			
			\node [cuteclosedswitchshape] (sw3) at ($(sw2) - (0,1)$) {};
			\draw (sw3.out) to[tDo] ++(1,0) to[short, -*] ++(1,0);
			
			\node [cuteopenswitchshape] (sw4) at ($(sw3) - (0,1)$) {};
			\draw (sw4.out) to[tDo] ++(1,0) to[short, -*] ++(1,0);
			
			% routing to main block
			\draw (sw1.in) -- ++(-1, 0) -- ++(0,-3) -- ++(1,0);
			\draw (sw2.in) to[short,-*] ++(-1, 0);
			\draw (sw3.in) to[short,-*] ++(-1, 0);
			
			% dashed block
			
			\draw ($(sw1.out) + (2,0)$) -- ++(0.5,0);
			\draw[dashed] ($(sw1.out) + (2.5,0)$) -- ++(1,0);
			\draw ($(sw1.out) + (3.5,0)$) -- ++(0.5,0);
			
			\draw ($(sw2.out) + (2,0)$) -- ++(0.5,0);
			\draw[dashed] ($(sw2.out) + (2.5,0)$) -- ++(1,0);
			\draw ($(sw2.out) + (3.5,0)$) to[short, -*] ++(0.5,0);
			
			\draw ($(sw3.out) + (2,0)$) -- ++(0.5,0);
			\draw[dashed] ($(sw3.out) + (2.5,0)$) -- ++(1,0);
			\draw ($(sw3.out) + (3.5,0)$) to[short, -*] ++(0.5,0);
			
			\draw ($(sw4.out) + (2,0)$) -- ++(0.5,0);
			\draw[dashed] ($(sw4.out) + (2.5,0)$) -- ++(1,0);
			\draw ($(sw4.out) + (3.5,0)$) -- ++(0.5,0) -- ++(0,3);
			
			%output
			\draw ($(sw3.out) + (4,0.5)$) to[short,*-] ++(1,0);
			
			%input
			\draw ($(sw3.in) + (-1,0.5)$) to[short,*-] ++(-1,0);
			
		\end{circuitikz}
	}};

	% Challenge
	\node[text=challengecolour,font=\scriptsize] (cval) at ($(circuit) + (0,1.3)$) {\texttt{0110\hspace{0.15em}1101\hspace{0.15em}....}};
	\node[text=challengecolour,font=\scriptsize] (ctext) at ($(cval) + (-0.2,0.25)$) {Challenge: $n$ bits};
	
	\node (boxstart) at ($(cval) - (1.1,-0.2)$) {};
	\draw[color=challengecolour, thick] (boxstart) -- ++(0,-0.4) -- ++ (2.2,0) -- ++ (0, 0.28);
	\draw[color=challengecolour, thick] ($(boxstart) + (0.75,-0.4)$) -- ++(0,0.1);
	\draw[color=challengecolour, thick] ($(boxstart) + (1.45,-0.4)$) -- ++(0,0.1);
	
	% Challenge map
	\draw[color=challengemapcolour, densely dotted, thick] ($(boxstart) + (0,-0.4)$) -- ++(-0.33,-0.4);
	\draw[color=challengemapcolour, densely dotted, thick] ($(boxstart) + (0.75,-0.4)$) -- ++(0.8,-0.4);
	\draw[color=challengemapcolour!50, densely dotted, thick] ($(boxstart) + (1.45,-0.4)$) -- ++(0.3,-0.2);
	\draw[color=challengemapcolour, densely dotted, thick] ($(boxstart) + (2.2,-0.4)$) -- ++(0.35,-0.4);
	
	% Resp graph
%	\node (rg) at (3,1) {\includegraphics[keepaspectratio,width=2cm]{images/ivchar-draft}};
	\node (rg) at (4,0.5) {\includegraphics[keepaspectratio,width=3.5cm]{images/VI-png}};
	
	% Response
	\node[text=respcolour] (rval) at ($(rg) + (1,-1.3)$) {\texttt{u8}||\texttt{u8}||...||\texttt{u8}};
	\node[text=respcolour,font=\scriptsize] (rtext) at ($(rval) - (0,0.26)$) {Response: $m$ bits};
	
	\draw[color=respcolour, dashed] ($([xshift=-25]rg.south) + (0,0.4)$) -- ($([xshift=-25]rg.south) + (0,1.3)$);
	\draw[color=respcolour, dashed] ($([xshift=14]rg.south) + (0,0.4)$) -- ($([xshift=14]rg.south) + (0,1.3)$);
	\draw[color=respcolour, <->] ($([xshift=-25]rg.south) + (0,0.7)$) -- ($([xshift=14]rg.south) + (0,0.7)$);
	
	\draw[->, color=respcolour] ($(rg) - (2.45,0.5)$) to[out=90, in=180] (rg.west);
	
	\draw[->, color=respcolour] ($([xshift=14]rg.south) + (0,0.4)$) to[out=-90, in=90] ([xshift=20,yshift=5]rval);
	\draw[->, color=respcolour] ($([xshift=-25]rg.south) + (0,0.7)$) to[out=-90, in=100] ([xshift=-20,yshift=5]rval);
\end{tikzpicture}}
%			\end{figure}
%		\end{column}
%	\end{columns}
%\end{frame}

\section{Q3: Performance}

\begin{frame}{Setup}
	\begin{columns}
		\begin{column}{0.5\linewidth}
			\textbf{Baselines}\\
			
			\vspace{1em}
			{\centering
			\resizebox{0.7\linewidth}{!}{\begin{tabular}{cc}
				\toprule
				Non-Polling&Polling\\
				\midrule
				\ourtech{} (XDP)&\ourtech{} (all)\\
				\ourtech{} (\afxdp)&\afp{}\\
				\ourtech{} (Split)&DPDK (NUC)\\
				\bottomrule
			\end{tabular}}}
			\vspace{1em}
			
			\textbf{Machines}
			\begin{itemize}
				\item Raspberry Pi Model 3B (\qty{100}{\mega\bit\per\second}),
				\item Intel i7 NUCs (\qty{1}{\giga\bit\per\second}).
			\end{itemize}
		\end{column}
		\begin{column}{0.5\linewidth}
			\textbf{NFs}
			\begin{itemize}
				\item Macswap,
				\item Blocking workloads ($\le$\qty{1}{\milli\second}).
			\end{itemize}
		
			\textbf{Why?}
			\begin{itemize}
				\item Power Draw on Pi, Latency/Throughput for all.
				\item Different architectures.
			\end{itemize}
		\end{column}
	\end{columns}
\end{frame}

%\begin{frame}{?? Results?}
%	content...
%\end{frame}

\begin{frame}{High-level Results}
	\begin{columns}
		\begin{column}{0.5\linewidth}
			\begin{itemize}
				\item Pure XDP \& \afxdp{} more CPU-efficient than polling baselines (line-rate on NUC).
				\item On RPi? Better than \afp{} on all metrics \alert{without polling}.
				\begin{itemize}
					\item Limited by fused Eth+USB controller.
				\end{itemize}
				\item XDP-Userland split prevents packet stalls with (conditionally) heavy chains.
				\begin{itemize}
					\item Limited by fused Eth+USB controller.
				\end{itemize}
			\end{itemize}
		More detail? \alert{Please check out our paper!}
		\end{column}
		\begin{column}{0.5\linewidth}
			\begin{figure}
				\centering
				\includegraphics[keepaspectratio,width=0.9\linewidth]{../plots/build/latency-vs-baselines/rpi-64B-trimlog-new}
				\caption{\emph{RPi} \qty{64}{\byte} packet latencies.\label{fig:lat-rpi}}
			\end{figure}
		\end{column}
	\end{columns}

\end{frame}

%\begin{frame}{...What's next?}
%	\begin{itemize}
%		\item Currently measuring on RPi and NUC:
%		\begin{itemize}
%			\item Power, CPU use, ...
%			\item Latency (distribution), Throughput
%			\item Showing usefulness in relocating `expensive' NFs.
%		\end{itemize}
%		\item Working out the details on paper for control plane reconfiguration:
%		\begin{itemize}
%			\item eBPF ProgMaps, etc. allow atomic replacement.
%			\item Still need to codify details on chain \& map building to prevent inconsistencies.
%		\end{itemize}
%	\end{itemize}
%\end{frame}

\begin{frame}[standout]
	Takeaways:
	\begin{itemize}
		\item \alert{Cheap NFs at the edge}: SBCs for packet processing.
		\item \alert{Low-latency and fast}: XDP path for majority of traffic, early \& cheap anomaly checks, power savings.
		\item \alert{Secure}: Rust NFs means memory safety \emph{and} performant.
		\item \alert{Easy to write}: \emph{native and XDP} portable NFs in Rust.
%		\item \emph{Ongoing work}: complex NFs, power + latency measures, better characterising PUF behaviour.
	\end{itemize}
	
	\vspace{2em}
	\alert{Questions?}\\
	{
		\scriptsize
		\vspace{2em}\faEnvelopeOpen{} \href{mailto:\myemail}{\nolinkurl{\myemail}}\\
		\vspace{-0.8em}	\small{\faGithub{} \href{https://github.com/\mygithub}{\mygithub} \hspace{0.5em} \faGlobe{} \url{\myurl}}
	}
	
	\begin{tikzpicture}[overlay, remember picture]
		%		\node[above right=0.8cm and 0.9cm of current page.south west] (esnet-logo) {\includegraphics[width=2.75cm]{netlab-trim}};
		%		\node[right=1cm of esnet-logo] {\adjincludegraphics[height=2cm,trim={0 {.4\height} 0 {.05\height}},clip]{uofg}};
		\node[above left=0.2cm and 0.8cm of current page.south east] {\includegraphics[width=2.75cm]{branding/netlab-fulllogo}};
		\node[above right=0.35cm and 0.8cm of current page.south west] {\includegraphics[width=2.75cm]{branding/uofg-white}};
	\end{tikzpicture}
\end{frame}

%\begin{frame}[allowframebreaks]{References}
%	\printbibliography[heading=none]
%\end{frame}

\end{document}